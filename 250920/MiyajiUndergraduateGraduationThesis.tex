\documentclass[
  a4paper,
  12pt,
  dvipdfmx
]{article}
\usepackage{titling}
\usepackage{amsmath,amssymb}
\usepackage{amsthm} %定理環境
\usepackage{bm}
\usepackage{url}
\usepackage[dvipdfmx]{graphicx, color}
\usepackage{ascmac}
\usepackage{enumerate} %箇条書き
\usepackage{enumitem}
\usepackage{mathtools}
\usepackage{amsfonts}
\usepackage{latexsym}
% \usepackage[all]{xy}
% \usepackage{ulem} %波線
\usepackage[normalem]{ulem}
% %\usepackage{eclbkbox}%四角枠
\usepackage{tocloft}%体裁を整える
\usepackage{titlesec}%見出しの設定
\usepackage{float}%図の位置
\usepackage{mathrsfs}%花文字
\usepackage{tikz}
\usepackage[dvipdfmx]{hyperref}
\usepackage{pxjahyper} % (u)pLaTeXのときのみかく
\usepackage{docmute} %分割に必要 
\usepackage{tikz-cd} %可換図式
\usepackage[margin=30truemm]{geometry}
% \usepackage{authblk}
\usetikzlibrary{arrows.meta}
\usetikzlibrary{patterns}
\usetikzlibrary{spath3}
\usetikzlibrary{knots}
\usetikzlibrary{hobby}
\usetikzlibrary{external}

%自分用のコマンド
\providecommand{\cprime}{\hbox{$'$}} %bibtex用
\newcounter{questionCounter}% 新しいカウンターを定義
\newcommand{\Qbox}{\stepcounter{questionCounter}\fbox{\thequestionCounter}}
\newcommand{\RR}{\mathbb{R}}
\newcommand{\CC}{\mathbb{C}}
\newcommand{\engname}[1]{(\textit{#1})}
\newcommand{\te}{\textrm{there exists\ }}
\newcommand{\st}{\ \textrm{such that}\ }
\newcommand{\symb}[3]{#1_{#2}^{#3}}
\newcommand{\Int}{\textrm{Int}}
\newcommand{\mcal}[1]{\mathcal{#1}}
\newcommand{\fall}{\textrm{for all}\ }
\newcommand{\suml}[2]{\sum\limits_{#1}^{#2}}
\newcommand{\soejiThree}[3]{#1_{{#2}_{#3}}}
\newcommand{\vecsigma}[1]{\overrightarrow{[\sigma_0][\sigma_1]},\overrightarrow{[\sigma_0][\sigma_2]},\ldots,\overrightarrow{[\sigma_0][\sigma_#1]}}
\newcommand{\ub}{\underline{b}}
\newcommand{\bb}{\bar{b}}
\newcommand{\sgn}{\mathrm{sgn}}
\newcommand{\simpangle}[1]{\langle#1\rangle}
\newcommand{\Ker}{\mathrm{Ker}\,}
\newcommand{\ZZ}{\mathbb{Z}}
\newcommand{\rank}{\mathrm{rank}}
\DeclareMathOperator{\gr}{gr} %graded
\DeclareMathOperator{\id}{id} %id
% \DeclareMathOperator{\Pr}{Pr} %Pr

\let\Re\relax%実部虚部の更新
\DeclareMathOperator{\Re}{Re}
\let\Im\relax
\DeclareMathOperator{\Im}{Im}

\title{
  \centerline{\mbox{卒論(タイトル未定)}}
  % Jones 多項式と Vassiliev 不変量の導入と関係性\\
  \large 元論文: Homomorphic expansions for knotted trivalent graphs
}
\author{宮路 宙澄}
\date{\today}
\hypersetup{
  colorlinks=false,
  pdfborder={0 0 1},
  linkbordercolor={red},
}

\theoremstyle{definition}
\renewcommand{\theenumi}{\roman{enumi}}
\renewcommand{\labelenumi}{(\theenumi)}
\renewcommand{\theenumii}{\theenumi-\alph{enumii}}
\renewcommand{\labelenumii}{(\theenumii)}
\renewcommand{\theenumiii}{\theenumii-\roman{enumiii}}
\renewcommand{\labelenumiii}{(\theenumiii)}
\renewcommand{\proofname}{証明}

\newcounter{mycounter} % 新しいカウンタを定義
\numberwithin{mycounter}{section} % mycounter を section と同期
\newtheorem{proposition}[mycounter]{Proposition} % mycounterカウンタを使用する
\newtheorem{theorem}[mycounter]{定理} % mycounterカウンタを使用する
\newtheorem{corollary}[mycounter]{Corollary}     % 同じカウンタを使う
\newtheorem{remark}[mycounter]{Remark}           % 同じカウンタを使う
\newtheorem{lemma}[mycounter]{Lemma}           % 同じカウンタを使う
\newtheorem{exercise}[mycounter]{Exercise}
\newtheorem{example}[mycounter]{例}
\newtheorem{definition}[mycounter]{Definition}
\begin{document}
\maketitle
\begin{abstract}
  KTGsに対し a universal Vassiliev invariant が存在することは知られていた\cite{murakami1997topological,cheptea2007tqft,dancso2010kontsevich}.
  KTGsにおいて``edge unzip''という操作のみ準同型にならず,補正項が現れる.dotted Knotted Trivalent Graphs において$Z^{old}$が準同型となるように$Z$を2通りで構成することが目的.
\end{abstract}

\tableofcontents
\newpage

\part{Introduction}
Knotted Trivalent Graphs のなす空間にはいい構造がある(Knots や links を含む).
4つの操作がある: orientation switch, edge delete, edge unzip, connected sum.
KTGs は有限生成である\cite{thurston2002algebra}.
%要確認
KTGsはKnot genus(ザイフェルト曲面?)やribbon property(ribbon knot?自己交差あり)などの良い代数構造をもつため,それらを使うことが出来る\cite{bar898algebraic}.

Knots の Kontsevich integral は universal Vassiliev invariant に拡張できる.
その中でも unzip 以外が準同型になる.
\begin{itemize}
  \item unzip, delete, connected sum を``tree connected sums''と呼ばれるより一般の操作へ変える.
  \item unzip が出来る edge を制限する.
\end{itemize}
簡単に$Z^{old}$をdKTGsで準同型にすることができ,dKTGsはKTGsの良い性質をすべて保つことを示す.
有限生成やclose connection to Drinfel'd associators (知らん) など.
\part{Preliminaries}
\section{KTGs and $Z^{old}$}

\begin{definition}
  \textit{Trivalent graph}とは,各頂点が3つの辺をもつグラフ.
\end{definition}

全ての辺は向きづけられているものとし,頂点は回転するように向きを与える.ループや円などの辺を許すこととする.

\begin{definition}
  \textit{Surface}とは第2可算公理(高々可算な開基を持つ)を満たす2次元多様体をいう.
\end{definition}

\begin{definition}
  $K$を単体複体,$\sigma, \tau$を以下の条件を満たす$K$の単体とする.
  \begin{itemize}
    \item $\tau \precneqq \sigma$,
    \item $\sigma$は$K$の最大の面単体で,他の最大面単体は$\tau$を含まない.このような$\tau$を\textit{free face}という.
  \end{itemize}
  このとき,$K$の\textit{collapse}とは,$\tau\preceq\gamma\preceq\sigma$となる$\gamma$をすべて取り除くことをいう.
\end{definition}

\begin{definition}
  単体複体$Y$における\textit{spine}とは,$Y$の部分単体複体$X$であって,$Y$をcollapseして$X$となるものをいう.
\end{definition}

\begin{definition}
  Trivalent graph $\Gamma$に対し,\textit{thickened trivalent graph} or \textit{framed trivalent graph}\cite{thurston2002algebra} $\mathbf{\Gamma}$とは,頂点を太らせたもの.参照先の定義では,1次元単体複体$\Gamma$と,surface $\Sigma$に対しそのspineとなるように$\Gamma$を埋め込んだもの(なめらか?)の組.
\end{definition}

\begin{definition}
  \textit{Knotted trivalent graph (KTG)}を,framed trivalent graph $\mathbf{\gamma}$から$\RR^3$への埋め込み,KTGの$\textit{skeleton}$をtrivalent graph $\Gamma$とする.(framed knotsやlinksも含む)
\end{definition}

KTGs において,skeleton が isotopy で移りあうものを同一視する.
Trivalent graph $\Gamma$ に対し,$\Gamma$をskeletonとする.すべてのKTGの集合を$\mathcal{K}(\Gamma)$と書く.

\begin{proposition}
  Framed knots と, knot diagrams で$R1', R2, R3$の操作で移りあうものを同一視したものは1対1に対応する.
\end{proposition}

\begin{proof}
  Quantum Invariants by 大槻 Theorem~1.8 p15
\end{proof}

\begin{proposition}
  KTGsのisotopy classとgraph diagrams (交点の上下の情報を残した射影)で$R1', R2, R3, R4$で移りあうものは1対1に対応する.
\end{proposition}
\begin{proof}
  \cite{dancso2010kontsevich} (p19下部) の [Y] An invariant of spatial graphs by Yamada,[Y] のp3のLemma~1.
\end{proof}

KTGsには以下の4つの操作がある.
\begin{definition}
  Trivalent graph $\Gamma$,KTGを$\gamma\in \mathcal{K}(\Gamma)$とし,$\Gamma$のedgeを$e$とする.$e$の\textit{switch the orientation}を,向きを変えるものとして定め,$S_e(\gamma)$と書く.
  \[S_e\colon \mathcal{K}(\Gamma)\to\mathcal{K}(S_e(\Gamma))\,;\,\gamma\mapsto S_e(\gamma)\]
\end{definition}

\begin{definition}
  $\Gamma$のedgeであって,両端に接続されているedgeの向きが一致している$e$を\textit{delete}するとは,$e$を削除し,三価性を保つように$e$の両端の頂点を削除することをいう.
  \[d_e\colon \mathcal{K}(\Gamma)\to\mathcal{K}(d_e(\Gamma))\,;\,\gamma\mapsto d_e(\gamma)\]
\end{definition}

\begin{definition}
  $\Gamma$のedge $e$を\textit{unzip}するとは,$e$を``限りなく近い''2つのedgesに分け,端点をなくすことをいう.端点をなくしたとき,edgeの向きが合っていることが必要である.
  同様の議論でframed graph $\mathbf{\Gamma}$に対し,unzipを定義できる.
  \[u_e\colon \mathcal{K}(\Gamma)\to\mathcal{K}(u_e(\Gamma))\,;\,\gamma\mapsto u_e(\gamma)\]
\end{definition}

\begin{definition}
  2つのtrivalent graph とそのedgeのペア$(\Gamma,e), (\Gamma',f)$の\textit{connected sum} $\Gamma\#_{e,f}\Gamma'$とは$e,f$をつなぐedgeを新たに作ること.well-definedであるために,新たなedgeの向きは$\Gamma$から$\Gamma'$への向きとし,KTGsにおいてはねじれを許さず,辺を付ける場合は$e,f$の右側に付けるとする.(2次元では自由に動かせないため左右が重要)
\end{definition}

KTGsのfinite type invariantsは,linksへのものを単に拡張する.同じskeletonのKTGsの形式和を許し,得られたベクトル空間を特異点の解消によってフィルター分けする.

\begin{definition}
  \textit{n-singular KTG}とは,$n$この特異点を持つtrivalent graphを$\RR^3$に埋め込んだもの.各特異点は横断的な2重点か,``$F$''と書かれた線上の点である.
\end{definition}

$n$-singular KTGの解消は,各2重点を上下の差とすることでとすることで得られる.また,``$F$-point''は,それを取り除いたframed graphと,1-unit分 twist させたframed graphの差として解消する.
%図

% この操作により$2^n$個のKTGsが現れる.

\newpage
\bibliographystyle{alpha} %plain, unsrt, alpha などが一般的
\bibliography{references} %bibファイルの指定
\end{document}