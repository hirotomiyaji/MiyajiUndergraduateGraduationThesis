%!TEX program = lualatex
\documentclass[aspectratio=169]{beamer}

\usetheme[block=fill, progressbar=foot]{moloch}

\usepackage{luatexja}
\usepackage{xcolor}
\usepackage{tcolorbox}
\tcbuselibrary{skins}
\definecolor{MyGreen}{RGB}{26, 109, 0}
% \setbeamercolor{palette primary}{bg=blue, fg=white}
\setbeamercolor{frametitle}{fg=white, bg=MyGreen!95}
\setbeamercolor{alerted text}{fg=red}
\setbeamercolor{block title}{bg=MyGreen!80, fg=white}
\setbeamercolor{block body}{bg=gray!20}
\setbeamercolor{progress bar}{fg=MyGreen}

\setbeamertemplate{block begin}{
  \begin{tcolorbox}[
    enhanced,              % 高度な描画機能を有効化
    title=\insertblocktitle,
    colframe=MyGreen!80,   % 枠(タイトル部)の色
    colback=gray!20,       % 本文の背景色
    coltitle=white,        % タイトルの文字色
    fonttitle=\bfseries,   % タイトルのフォント
    arc=2mm,               % 【ここを調整】角の丸み
    drop shadow,           % 【ここを調整】影を有効化
    boxrule=0.5mm,         % 枠線の太さ
    top=2mm, bottom=2mm, left=2mm, right=2mm % 内側の余白
  ]
}
\setbeamertemplate{block end}{\end{tcolorbox}}

\makeatletter
\setlength{\moloch@progressinheadfoot@linewidth}{3pt} % ここで太さを指定(デフォルトは0.4pt)
\makeatother


\title{宮路の卒論}
\author{宮路 宙澄}
\date{\today}

\begin{document}

\maketitle

\begin{frame}{目次}
  \tableofcontents
\end{frame}

\section{前提知識}

\begin{frame}{前提知識}
  \begin{definition}[KTG]
    \textit{Knotted trivalent graph (KTG)}を,framed trivalent graph $\mathbf{\gamma}$から$\mathbb{R}^3$への埋め込み,KTGの\textit{skeleton}をtrivalent graph $\Gamma$とする.
  \end{definition}
\end{frame}

\section{KTGs and $Z^{old}$}

\begin{frame}{研究の背景}
  ここに内容を記述します。
  \begin{itemize}
    \item 従来の結び目理論では...
    \item 本研究の目的は...
  \end{itemize}
  
  \vspace{1em}
  重要な点は \alert{ここです}。
\end{frame}

\begin{frame}{スライドのタイトル}
  頑張る
  \begin{itemize}
    \item あ
    \item い
  \end{itemize}
\end{frame}

\begin{frame}{定義と定理}
  数学の発表ではブロックが見やすいと便利です。

  \begin{definition}[結び目]
    つながった紐
  \end{definition}

  \begin{theorem}[主要定理]
    $1+1=2$.
  \end{theorem}
\end{frame}

\end{document}