\documentclass[
  a4paper,
  12pt,
  dvipdfmx
]{article}
\usepackage{titling}
\usepackage{amsmath,amssymb}
\usepackage{amsthm} %定理環境
\usepackage{bm}
\usepackage{url}
\usepackage[dvipdfmx]{graphicx, color}
\usepackage{ascmac}
\usepackage{enumerate} %箇条書き
\usepackage{enumitem}
\usepackage{mathtools}
\usepackage{amsfonts}
\usepackage{latexsym}
% \usepackage[all]{xy}
% \usepackage{ulem} %波線
\usepackage[normalem]{ulem}
% %\usepackage{eclbkbox}%四角枠 
\usepackage{tocloft}%体裁を整える
\usepackage{titlesec}%見出しの設定
\usepackage{float}%図の位置
\usepackage{mathrsfs}%花文字
\usepackage{tikz}
\usepackage[dvipdfmx]{hyperref}
\usepackage{pxjahyper} % (u)pLaTeXのときのみかく
\usepackage{docmute} %分割に必要 
\usepackage{tikz-cd} %可換図式
% \usepackage[margin=30truemm]{geometry}
% \usepackage{authblk}
\usetikzlibrary{arrows.meta}
\usetikzlibrary{patterns}
\usetikzlibrary{spath3}
\usetikzlibrary{knots}
\usetikzlibrary{hobby}
\usetikzlibrary{external}

%自分用のコマンド
\providecommand{\cprime}{\hbox{$'$}} %bibtex用
\newcounter{questionCounter}% 新しいカウンターを定義
\newcommand{\Qbox}{\stepcounter{questionCounter}\fbox{\thequestionCounter}}
\newcommand{\RR}{\mathbb{R}}
\newcommand{\CC}{\mathbb{C}}
\newcommand{\engname}[1]{(\textit{#1})}
\newcommand{\te}{\textrm{there exists\ }}
\newcommand{\st}{\ \textrm{such that}\ }
\newcommand{\mcal}[1]{\mathcal{#1}}
\newcommand{\fall}{\textrm{for all}\ }
\newcommand{\suml}[2]{\sum\limits_{#1}^{#2}}
\newcommand{\simpangle}[1]{\langle#1\rangle}
\newcommand{\ZZ}{\mathbb{Z}}
\newcommand{\NN}{\mathbb{N}}
\newcommand{\QQ}{\mathbb{Q}}
\newcommand{\ca}{\mathcal{A}}
\DeclareMathOperator{\Ker}{Ker}
\DeclareMathOperator{\rank}{rank}
\DeclareMathOperator{\gr}{gr} %graded
\DeclareMathOperator{\id}{id} %id
% \DeclareMathOperator{\Pr}{Pr} %Pr

\let\Re\relax%実部虚部の更新
\DeclareMathOperator{\Re}{Re}
\let\Im\relax
\DeclareMathOperator{\Im}{Im}

\title{
  \centerline{\mbox{卒論(タイトル未定)}}
  % Jones 多項式と Vassiliev 不変量の導入と関係性\\
  \large 元論文: Homomorphic expansions for knotted trivalent graphs
}
\author{宮路 宙澄}
\date{\today}
\hypersetup{
  colorlinks=false,
  pdfborder={0 0 1},
  linkbordercolor={red},
}

\theoremstyle{definition}
\renewcommand{\theenumi}{\roman{enumi}}
\renewcommand{\labelenumi}{(\theenumi)}
\renewcommand{\theenumii}{\theenumi-\alph{enumii}}
\renewcommand{\labelenumii}{(\theenumii)}
\renewcommand{\theenumiii}{\theenumii-\roman{enumiii}}
\renewcommand{\labelenumiii}{(\theenumiii)}
% \renewcommand{\proofname}{証明}

\newcounter{mycounter} % 新しいカウンタを定義
\numberwithin{mycounter}{section} % mycounter を section と同期
\newtheorem{proposition}[mycounter]{Proposition} % mycounterカウンタを使用する
\newtheorem{theorem}[mycounter]{定理} % mycounterカウンタを使用する
\newtheorem{corollary}[mycounter]{Corollary}     % 同じカウンタを使う
\newtheorem{remark}[mycounter]{Remark}           % 同じカウンタを使う
\newtheorem{lemma}[mycounter]{Lemma}           % 同じカウンタを使う
\newtheorem{exercise}[mycounter]{Exercise}
\newtheorem{example}[mycounter]{例}
\newtheorem{definition}[mycounter]{Definition}
\begin{document}
\maketitle
\begin{abstract}
  KTGsに対し a universal Vassiliev invariant が存在することは知られていた\cite{murakami1997topological,cheptea2007tqft,dancso2010kontsevich}.
  KTGsにおいて``edge unzip''という操作のみ準同型にならず,補正項が現れる.dotted Knotted Trivalent Graphs において$Z^{old}$が準同型となるように$Z$を2通りで構成することが目的.
\end{abstract}

\tableofcontents
\newpage

\part{Introduction}
Knotted Trivalent Graphs(Knots や links を含む) のなす空間には良い構造がある.
次の4つの操作がある: orientation switch, edge delete, edge unzip, connected sum.
KTGs は有限生成である\cite{thurston2002algebra}.
%要確認
KTGsはKnot genus(ザイフェルト曲面?)やribbon property(ribbon knot?自己交差あり)などの良い代数構造をもつため,それらを使うことが出来る\cite{bar898algebraic}.

Knots の Kontsevich integral は universal Vassiliev invariant に拡張できる.
その中でも unzip 以外が準同型になる.
\begin{itemize}
  \item unzip, delete, connected sum を``tree connected sums''と呼ばれるより一般の操作へ変える.
  \item unzip が出来る edge を制限する.
\end{itemize}
簡単に$Z^{old}$をdKTGsで準同型にすることができ,dKTGsはKTGsの良い性質をすべて保つことを示す.
有限生成やclose connection to Drinfel'd associators (知らん) など.
\part{Preliminaries}
\section{KTGs and $Z^{old}$}

\begin{definition}
  \textit{Trivalent graph}(3価グラフ)とは,各頂点が3つの辺をもつようなグラフをいう.
\end{definition}

全ての辺は向きづけられているものとし,頂点は反時計回りに向きを与える.ループや円などの辺を許すこととする.

\begin{definition}
  \textit{Surface}(曲面)とは第2可算公理\footnote{高々可算な開基を持つ.}を満たす2次元多様体をいう.
\end{definition}

\begin{definition}
  $K$を単体複体,$\sigma, \tau\in K$を以下の条件を満たすとする.
  \begin{itemize}
    \item $\tau \precneqq \sigma$,
    \item $\sigma$は$K$の最大の面単体で,他の最大面単体は$\tau$を含まない.このような$\tau$を\textit{free face}という.
  \end{itemize}
  このとき,$K$の\textit{collapse}とは,$\tau\preceq\gamma\preceq\sigma$となる$\gamma$をすべて取り除くことをいう.

  \raisebox{-0.4\height}{\includegraphics[scale=0.5]{images/collapseNot.png}}

  \raisebox{-0.4\height}{\includegraphics[scale=0.5]{images/collapse.png}}
\end{definition}

\begin{definition}
  単体複体$K$における\textit{spine}(スパイン)とは,$K$の部分単体複体$K'$であって,$K$をcollapseして$K'$となるものをいう.
\end{definition}

\begin{definition}
  graph $\Gamma$に対し \textit{framed graph} (枠付きグラフ) $\mathbf{\Gamma}$とは,$\Gamma$と,$\Gamma$をspineとして曲面$\Sigma$へ埋め込む\footnote{グラフを1次元CW複体とみなす.}写像$\Gamma\hookrightarrow\Sigma$の組をいう.特に$\Gamma$がTrivalent graphのとき,$\mathbf{\Gamma}$を\textit{framed trivalent graph}\footnote{論文では thickened trivalent graph と書いてある.}という.
  \begin{figure}[H]
    \centering
    \raisebox{-0.4\height}{\includegraphics[scale=0.3]{inkscape/trivalentgraph.pdf}}
    $\leadsto$
    \raisebox{-0.4\height}{\includegraphics[scale=0.23]{inkscape/framedtrivalentgraph.pdf}}
  \end{figure}
\end{definition}

% \begin{definition}
%   Trivalent graph $\Gamma$に対し,\textit{framed trivalent graph}\footnote{論文には thickened trivalent graph と書いてある.}(枠付き3価グラフ)\cite{thurston2002algebra} $\mathbf{\Gamma}$とは,頂点を太らせたもの.参照先の定義では,1次元単体複体$\Gamma$と,surface $\Sigma$に対しそのspineとなるように$\Gamma$を滑らかに埋め込んだものの組.
% \end{definition}

\begin{definition}
  \textit{Knotted trivalent graph (KTG)}を,framed trivalent graph $\mathbf{\gamma}$から$\RR^3$への埋め込み,KTGの$\textit{skeleton}$をtrivalent graph $\Gamma$とする.(framed knotsやlinksも含む)
  \begin{figure}[H]
    \centering
    \raisebox{-0.4\height}{\includegraphics[scale=0.3]{inkscape/exKTG.pdf}}
    \caption{Knotted Trivalent Graphの例}
    \label{fig:exKTG}
  \end{figure}
\end{definition}

KTGs において,skeleton が isotopy で移りあうものを同一視する.特に,framed knots や links は KTGs の特別な場合である.
Trivalent graph $\Gamma$ に対し,すべてのKTGの集合を$\mathcal{K}(\Gamma)$と書き,$\mathcal{K}\coloneqq \bigcup_{\Gamma}\mathcal{K}(\Gamma)$とする.

\begin{proposition}
  Framed knots と,framed knot の diagrams で 3つの Reidemeister 変形 $R1', R2, R3$ の操作で移りあうものを同一視したものは1対1に対応する.
  \begin{figure}[H]
    \centering
    $R1'$:\hspace{3mm}
    \raisebox{-0.45\height}{\includegraphics[scale=0.4]{inkscape/FramedReideMoves/fR1_0.pdf}} \hspace{3mm}$\longleftrightarrow$ \hspace{3mm}
    \raisebox{-0.45\height}{\includegraphics[scale=0.4]{inkscape/FramedReideMoves/fR1_1.pdf}} \hspace{3mm}$\longleftrightarrow$\hspace{3mm}
    \raisebox{-0.45\height}{\includegraphics[scale=0.4]{inkscape/FramedReideMoves/fR1_2.pdf}}
    
    \vspace{3mm}
    $R2$:\hspace{3mm}
    \raisebox{-0.45\height}{\includegraphics[scale=0.4]{inkscape/FramedReideMoves/fR2_0.pdf}} \hspace{3mm}$\longleftrightarrow$ \hspace{3mm}
    \raisebox{-0.45\height}{\includegraphics[scale=0.4]{inkscape/FramedReideMoves/fR2_1.pdf}} \hspace{3mm}$\longleftrightarrow$\hspace{3mm}
    \raisebox{-0.45\height}{\includegraphics[scale=0.4]{inkscape/FramedReideMoves/fR2_2.pdf}}
    
    \vspace{3mm}
    $R3$:\hspace{3mm}
    \raisebox{-0.45\height}{\includegraphics[scale=0.4]{inkscape/FramedReideMoves/fR3_0.pdf}} \hspace{3mm}$\longleftrightarrow$ \hspace{3mm}
    \raisebox{-0.45\height}{\includegraphics[scale=0.4]{inkscape/FramedReideMoves/fR3_1.pdf}}
    \caption{Framed knots における3種類の Reidemeister 変形}
    \label{fig:Reidemeister}
  \end{figure}
\end{proposition}

\begin{proof}
  \cite{ohtsuki2002quantum}~P15~Theorem~1.8
\end{proof}

\begin{proposition}
  KTGsのisotopy classとgraph diagrams (交点の上下の情報を残した射影)で$R1', R2, R3, R4$で移りあうものは1対1に対応する.

  \begin{figure}[H]
    \centering
    $R4a$:\hspace{3mm}
    \raisebox{-0.45\height}{\includegraphics[scale=0.4]{inkscape/FramedReideMoves/fR4a_0.pdf}} \hspace{3mm}$\longleftrightarrow$ \hspace{3mm}
    \raisebox{-0.45\height}{\includegraphics[scale=0.4]{inkscape/FramedReideMoves/fR4a_1.pdf}}

    \vspace{3mm}
    $R4b$:\hspace{3mm}
    \raisebox{-0.45\height}{\includegraphics[scale=0.4]{inkscape/FramedReideMoves/fR4b_0.pdf}} \hspace{3mm}$\longleftrightarrow$ \hspace{3mm}
    \raisebox{-0.45\height}{\includegraphics[scale=0.4]{inkscape/FramedReideMoves/fR4b_1.pdf}}
    \caption{Framed knots における3種類の Reidemeister 変形}
    \label{fig:ReidemeisterKTG}
  \end{figure}
\end{proposition}
\begin{proof}
  % \cite{dancso2010kontsevich} (p19下部) で引用されている.\cite{yamada1987invariant} のp3のLemma~1.
  % 「$\RR^3$に埋め込まれたグラフ (空間グラフ) が isotopic であることと,その diagrams が Reidemeister moves で変形できるということが同値.」だが,怪しい.一旦パス.
  % \cite{murakami1997topological}~P505~Theorem~1.4\\
  % 向きづけられた枠付き3価グラフ$\mathbf{\Gamma}$の$S^3$への埋め込み$G$の universal Vassiliev invariant $\hat{Z}(G)$ は isotopy invariant である.(Reidemeister moves で移り変わっても変わらない)
\end{proof}

KTGsには以下の4つの操作がある.
\begin{definition}
  Trivalent graph を$\Gamma$,KTGを$\gamma\in \mathcal{K}(\Gamma)$とし,$\Gamma$のedgeを$e$とする.$e$の\textit{switch the orientation}を,向きを変えるものとして定め,$S_e(\gamma)$と書く.
  \[S_e\colon \mathcal{K}(\Gamma)\to\mathcal{K}(S_e(\Gamma))\,;\,\gamma\mapsto S_e(\gamma)\]
  % \[S_e\colon \mathcal{K}\to\mathcal{K}\,;\,\gamma\mapsto S_e(\gamma)\]

  図
\end{definition}

\begin{definition}
  $\Gamma$のedgeであって,両端に接続されているedgeの向きが一致している$e$を\textit{delete}するとは,$e$を削除し,三価性を保つように$e$の両端の頂点を削除することをいう.
  \[d_e\colon \mathcal{K}(\Gamma)\to\mathcal{K}(d_e(\Gamma))\,;\,\gamma\mapsto d_e(\gamma)\]
  % \[d_e\colon \mathcal{K}\to\mathcal{K}\,;\,\gamma\mapsto d_e(\gamma)\]

  図
\end{definition}

\begin{definition}
  $\Gamma$のedge $e$を\textit{unzip}するとは,$e$を``限りなく近い''2つのedgesに分け,端点をなくすことをいう.端点をなくしたとき,edgeの向きが合っていることが必要である.
  同様の議論でframed graph $\mathbf{\Gamma}$に対し,unzipを定義できる.
  \[u_e\colon \mathcal{K}(\Gamma)\to\mathcal{K}(u_e(\Gamma))\,;\,\gamma\mapsto u_e(\gamma)\]
  % \[u_e\colon \mathcal{K}\to\mathcal{K}\,;\,\gamma\mapsto u_e(\gamma)\]

  図
\end{definition}

\begin{definition}
  2つのtrivalent graph とそのedgeのペア$(\Gamma,e), (\Gamma',f)$の\textit{connected sum} $\Gamma\#_{e,f}\Gamma'$とは$e,f$をつなぐedgeを新たに作ること.well-definedであるために,新たなedgeの向きは$\Gamma$から$\Gamma'$への向きとし,KTGsにおいてはねじれを許さず,辺を付ける場合は$e,f$の右側に付けるとする.(2次元では自由に動かせないため左右が重要)
  \[\#_{e,f}\colon \mathcal{K}\to \mathcal{K}\,;\,\]

  図
\end{definition}

KTGsのfinite type invariantsは,linksへのものを単に拡張する.同じskeletonのKTGsの形式和を許し,得られたベクトル空間を特異点の解消によってフィルター分けする.

\[\mathcal{F}^0(\Gamma)\coloneqq\left\{\sum_{i=1}^m a_i \mathbf{\Gamma}\middle| m\in\NN, a_i\in \QQ, \Gamma_i\in \mathcal{K}(\Gamma)\right\}\]
をKTGsの有限な形式和全体がなす$\QQ$-ベクトル空間とする.

\begin{definition}
  \textit{n-singular KTG}とは,$n$この特異点を持つtrivalent graphの$\RR^3$へのはめ込み.各特異点は横断的な2重点か,``$F$''と書かれた線上の点である.
\end{definition}
$n\geq 1$に対し以下のようなベクトル空間を考える.
\[\mathcal{F}^n(\Gamma)\coloneqq\left\{\sum_{i=1}^m a_i \mathbf{\Gamma}\middle| m\in\NN, a_i\in \QQ, \Gamma'_i\colon \Gamma\text{を骨格とし少なくとも}n\text{この特異点を持つKTG}\right\}\]
特異点を解消する写像として,
\[\rho\colon\mathcal{F}^\ast \to \mathcal{F}^0,\]
であって,以下のように定めるものを考える:
\[\text{図}\]
$\mathcal{F}^n (n\geq 1)$に対し,$\rho$で送ると,
\[\mathcal{F}^0\supset \rho(\mathcal{F}^1)\supset \rho(\mathcal{F}^2)\cdots.\]
というfiltrationが得られる.このfiltrationに対するassociated graded spaceとして,
\[\mathcal{A}(\Gamma)\coloneqq \prod_{i=0}^{\infty}\mathcal{F}^{i}(\Gamma)/\mathcal{F}^{i+1}(\Gamma)\]
とする.
% $n$-singular KTGの解消は,各2重点を上下の差とすることでとすることで得られる.また,``$F$-point''は,それを取り除いたframed graphと,1-unit分 twist させたframed graphの差として解消する.
% 図
% この操作により$2^n$個のKTGsが現れる.
% Vassiliev invariantを構成する段階で現れるfiltration\footnote{付録にて解説} $\mathcal{F}^n(\Gamma)$は$n$-singular KTGs (skeletonは$\Gamma$)を解消して得られたものの線形和で表される.
% \[\mathcal{A}(\Gamma)= \bigoplus_{n=0}^\infty \mathcal{F}^n(\Gamma)/\mathcal{F}^{n+1}(\Gamma)\ (\mathcal{F}^0(\Gamma)=\mathcal{K}(\Gamma))\]
% を(filtrationに対する)\textit{associated graded space}とし(vector spaceである.要証明),$\mathcal{A}\coloneqq \bigcup_\Gamma\mathcal{A}(\Gamma)$とする.

$\mathcal{A}(\Gamma)$はchord diagramを用いて表すことができる.

\begin{definition}
  skeleton graph $\Gamma$上の$n$次の\textit{chord diagram}とは,$\Gamma$の辺上の$2n$この点のペアからなる組み合わせ的なものであり,辺の向きを保つ同相写像で移りあうものを同一視する.特に,次数$n$のchord diagramを基底とする$\QQ$上のベクトル空間を$\mathcal{D}_n(\Gamma)$と書く.
\end{definition}

% $D_1, D_2\in \mathcal{D}_n(\Gamma)$に対し,$D_1\sim_{\text{Diag}} D_2$を

\begin{proposition}
  $\pi\colon\mathcal{D}(\Gamma)\to \mathcal{A}(\Gamma)$ は well-defined であり surjective.

  aaaaasadafa
\end{proposition}

\begin{proof}
  この論文だけ$\ca(\Gamma)$の構成法が違うので一旦パス.(結局は$\{\text{Chord diagrams}\}/(\text{VI, 4T})\cong \ca$のはず)
\end{proof}

上記の写像の kernel に含まれる2つの関係式がある.(要証明←goodnoteとAlge Knot Theoryのノートでできたはず)(Kerが一致するとはまだ言ってない)

\begin{itemize}
  \item (4T) Four term relation
  図
  \item (VI) Vertex invariance relation
  図
\end{itemize}
図に描かれていない部分にはgraphがあるが,それらは全て同じでなければならない.4Tでは反時計回りの向きを与える(これ必要?).VI において,$(-1)^\to$は,chord の付いたedgeが外向きなら-1, 内向きなら1をかける(つまり式は8つある).

4T, VI のrelationsが存在することは分かったが,これ以上のrelationが存在``しない''ことを示すのは困難である.これを示すには,universal finite type invariant $\QQ \text{KTG}\to \ca$を構成するのが最善である(ここでは定義しないが,後で一般の文脈で定義する).これは,T.Le, H.Murakami, J. Murakami, T.Ohtsuki の結果をもとに,またDrinfeldのassociatorの理論を用いて[KO, CD],\cite{bar898algebraic}でのKontsevich integral を拡張する形で\cite{murakami1997topological}で初めて得られた.

KTGsの各operationは$\ca$上のoperationを誘導する.($\ca$は$\mathcal{K}(\Gamma)$のassociated graded spaceである.)

\begin{itemize}
  \item orientation switch
  \item edge delete
  \item edge unzip
  \item connected sum
  well-definedである.Introduction to Vassiliev knot invariants(Chmutov) のLemma4.2.9

\end{itemize}

\newpage
\bibliographystyle{alpha} %plain, unsrt, alpha などが一般的
\bibliography{references} %bibファイルの指定
\end{document}